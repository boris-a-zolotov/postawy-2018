\documentclass[a4paper]{extarticle}
\usepackage{xamprogram}

\renewcommand{\kurs}{«Преобразования и перестановки», 8 класс}

\begin{document}
\ \\ [-0.7cm]

\begin{enumerate}

\itm Ассоциативность композиции функций.

\itm Группа, подгруппа. Подгруппа — то же, что замкнутое множество.

\itm Группа $\Bij(X)$. Изоморфность таких групп в случае равномощных множеств.

\itm Сопряжённость элементов — отн. экв. Изоморфность сопряжённых подгрупп.

\itm Сопряжение элементом из подгруппы оставляет подгруппу на месте. Нормальная под- группа\scolon $g^{-1}Hg = H$ $\Leftrightarrow$ $gH=Hg$.

\itm Группа $S_n$. Поиск обратной, композиция перестановок, сопряжение. Теорема: сопряжение $x$ посредством $g$ — то же самое, что подействовать $g$ на запись $x$.

\itm Ядро гомоморфизма. Пример, когда ядро нетривиально. Ядро гомоморфизма — под- группа\scolon нормальная подгруппа.

\itm Разложение перестановки на циклы. Различные записи циклов. Непересекающиеся циклы коммутируют, пересекающиеся циклы не коммутируют. Транспозиции, разложение цикла на транспозиции, разложение любой перестановки на транспозиции.

\itm Количество цикл.\,типов с двумя циклами. Количество перестановок данного цикл.\,типа.

\itm Обратная перестановка для цикла. С какими перестановками коммутирует $\sigma(x) = x+1$.

\itm Сколько циклов в разложении $\sigma^m$.

\itm Сопряженность перестановок $\Longleftrightarrow$ один цикленный тип.

\itm Инверсии, количество инверсий, знак перестановки. Определение знака перестановки через произведение знаков каких-то чисел. Знак транспозиции.

\itm А-инверсии, $\pi$-инверсии, однозначное соответствие между ними и инверсиями. Эквивалентность разных определений знака. 

\itm Знак — гомоморфизм $S_n \longrightarrow \{ -1, 1\}$.

\itm Подгруппа $A_n \leqslant S_n$, её нормальность, её порядок. Разложение на транспозиции и чётность перестановки.

\itm Теорема Кэли — доказательство.

\itm Группа преобразований множества $X$ — определение. $\Iso (\b R^2)$ — определение. Доказательство того, что изометрии биективны. Доказательство того, что $\Iso (\b R^2) \leqslant \Bij (\b R^2)$. Коммутативна ли $\Iso (\b R^2)$?

\itm Группа $V(\b R^2)$. Нормальность $V(\b R^2) \leqslant \Iso (\b R^2)$ — доказательство двумя способами.

\itm Циклическая группа: количество элементов, коммутативность, порождающий элемент, изоморфность $\b Z_n$, вложение в $S_n$ явным образом.

\itm Диэдральная группа: количество элементов, некоммутативность, образующие и соотношения, вложение в $S_{2n}$ и $S_n$.

\itm Реализация $C_n$ и $D_n$ как вращений пирамиды и диэдра.

\itm Группа $\SO_2$, её элементы. Теорема о конечных подгруппах $\SO_2$.

\itm Группа $\O_2$, её элементы. Группы $\O_2^{(0,0)}$ и $\O_2^{(x,y)}$ сопряжены.

\itm Теорема о конечных подгруппах $\Iso (\b R^2)$.








\end{enumerate}
\end{document}