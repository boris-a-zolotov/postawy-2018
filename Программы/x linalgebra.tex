\documentclass[a4paper]{extarticle}
\usepackage{xamprogram}
\renewcommand{\kurs}{«Линейная алгебра», 10 класс}

\DeclareMathOperator{\Hom}{Hom}

\begin{document}
\begin{enumerate}

\itm Определение модуля. $(-1)x = -x$, $0x = 0$. Примеры: $_RR$, $\{0\}$, $_{\b Z} G$, $_{\b Z}\b Z_n$, $_RR^n$, $_RR^\infty$, $_RR^X$, $\oplus$.

\itm Линейные отображения, основные типы, примеры: $0$, $x \mapsto \lambda x$, вложение и проекция в $\b Z$--модулях. Подмодули, примеры: $V$, $\{0\}$, постоянные функции, подгруппа для группы, многочлены степени $\le n$. Подмодули на плоскости $\b R^2$.

\itm Пересечение и сумма модулей как $\sup$ и $\inf$.

\itm Ядро и образ: почему подмодули, как характеризуют отображение. График как подмодуль $\oplus$.

\itm Фактор-модуль: отношение $\sim_{_V}$, корректность определения операций.

\itm Вложения и проекции для внешней прямой суммы. Их композиции.

\itm Три теоремы о гомоморфизме.

\itm Линейная оболочка векторов как наименьший подмодуль, который их содержит.

\itm Внутренняя прямая сумма, однозначность представления, изоморфизм с внешней прямой суммой.

\itm Точная последовательность. Два примера точных последовательностей, связанных с гомоморфизмом~$\varphi$ (с коядром, без коядра). Точная последовательность из суммы, прямой суммы и пересечения.

\itm Линейная независимость, линейная независимость бесконечных систем. Базис, теорема: базис $\Leftrightarrow$ у любого вектора единственное представление.

\itm Формальные линейные комбинации, $R^{(X)} \leqslant R^X$. Матрицы и многочлены как модули формальных линейных комбинаций.

\itm Модуль с базисом $u_1 \ldots u_n$ изоморфен $R^n$. Инъективный гомоморфизм сохраняет линейную независимость, сюръективный сохраняет порождающесть.

\itm Два свободных модуля изоморфны $\Longleftrightarrow$ в них есть базисы одного размера.

\itm Умножение матриц. Ассоциативность умножения матриц.

\itm Неединственность ранга: кольцо матриц, где $R^2 \cong R$.

\itm Матрица перехода от базиса к базису: как её построить, свойства матриц перехода. Как преобразуются строчки из векторов, как преобразуются столбцы из координат.

\itm Матрица линейного отображения: как её построить, преобразование матрицы линейного отображения при переходе между базисами.

\itm Линейная зависимость над полем, следствия про линейные оболочки. Условие замены Штейница.

\itm Теорема Штейница, доказательство методом замены.

\itm Базис как минимальная порождающая и максимальная линейно независимая система.

\itm Существование и равномощность базисов. Размерность, размерность прямой суммы. Векторные пространства одной размерности изоморфны.

\itm Дополнение системы до линейно независимой через вектор с неизвестной координатой: общий метод.

\itm Обратная матрица $2 \times 2$, условие её существования.

\itm Относительный базис, четыре эквивалентных определения\scolon коразмерность.

\itm Теорема о размерности ядра и образа, теорема о размерности суммы и пересечения.

\itm Выделение подмодуля, заданного условием, его размерность, базис и относительный базис: примеры.

\itm Структура векторного пространства на $\Hom (U,V)$. Функториальность для $\Hom(U,V)$.

\itm Базис в пространстве $\Hom(U,V)$: линейная независимость, порождающесть. $\Hom(K,V) = V$.

\itm Двойственное пространство. Двойственный базис. Преобразование столбцов базисов в $V^*$ и строчек координат при переходе между базисами $V$.




\end{enumerate}
\end{document}