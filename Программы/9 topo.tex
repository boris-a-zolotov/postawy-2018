\documentclass[a4paper,11pt]{extarticle}
\usepackage{xamprogram}
\renewcommand{\kurs}{«Общая топология», 9 класс}

\newgeometry{left=0.3cm,right=0.87cm,
	top=0.35cm,bottom=0.3cm}

\begin{document}

\begin{enumerate}

\itm $ff^{-1} (B) \subseteq B$\scolon $f^{-1}f (A) \supseteq A$.

\itm Определение счётного множества. Счётность множеств $\b Q$, $\b Z$, $\b N^k$, $\mathrm{Seq^f}\,\b N$. Счётность счётного объединения счётных множеств, декартова произведения счётных множеств.

\itm $\EuScript P(M)$ не равномощно $M$. Континуальные множества: отрезок, $\b R$, $\mathrm{Seq}\,\mathbb N$. Континуальность $E$.

\itm Канторово множество: описание через троичную запись, «длина» 0, замкнутость.

\itm Топологическое пространство: определение. Примеры: $(\b R, \O_{std})$, $(\b R, \mathcal S)$\scolon дискретная и антидискретная топологии. Прямая Зоргенфрея, плоскость Немыцкого — построение, почему это не $\O_{std}$. Окрестность точки, открытое множество содержит каждую свою точку вместе с окрестностью.

\itm База топологии. Базовое множество в открытом $U$, содержащее данную $x \in U$. Два свойства базы. Счётная база в $(\b R, \O_{std})$, в $(\b R, \mathcal S)$. Базы в дискретных и антидискретных топологиях. Предбаза.

\itm База в точке. Построение $\B \rar \forall x\,\B(x)$\scolon $\forall x\,\B(x) \rar \B$. Аксиомы счётности\scolon пространство с первой, но без второй аксиомы счётности.

\itm Топологии Зарисского. Аксиомы счётности в $\O^{\b R}_{\mathrm{cf}}$, $\O^{\b R}_{\mathrm{cc}}$, $\O^{\b N}_{\mathrm{cf}}$, $\O^{\b N}_{\mathrm{cc}}$.

\itm Замкнутые множества\scolon их свойства, двойственные открытым. Отрезок замкнут. Замыкание множества. Монотонность замыкания.

\itm Замыкания $\b Q$ и $(a,b)$ в $\O_{std}$ и в $\mathcal S$. Замыкание пустого, замыкание объединения, замыкание замыкания. Критерий принадлежности $x \in \Cl A$.

\itm Внутренность множества. Монотонность внутренности. Связь внутренности и замыкания ($X \setminus \ldots$). Внутренность $\b Q$ в $\O_{std}$.

\itm Внутренность множества. $\Int X$, $\Int (A \cap B)$, $\Int \Int A$. Критерий принадлежности $x \in \Int A$.

\itm Применение операций $\Cl$ и $\Int$. $\Cl \Int \Cl \Int = \Cl \Int$ ($\Int \Cl \Int \Cl = \Int \Cl$). Когда все семь „доступных“ множеств различны (на примере $\O_{std}$).

\itm $\O_{x_0}$, почему это топология. База этой топологии, замыкания и внутренности в этой топологии. Стандартная топология окружности, стандартная топология отрезка.

\itm Индуцированная топология, топология дизъюнктного объединения. Топология на двух отрезках как индуцированная и как дизъюнктное объеднение. Компоненты связности в этой топологии.

\itm $(\b R^2, \O_{std})$. Она индуцирует $\O_{std}$ на $\b R$, замкнутые подмножества прямой сохраняются. Топология и графы: примеры открытых множеств, компоненты связности.

\itm Граница множества. $\Int (\partial A)$, если $A$ открыто, — пуста. Пример $X \subset \b R$ такого, что $\partial X = \b R$. Критерий принадлежности $x \in \partial A$.

\itm Граница и теоретико-множественные операции: 9 свойств.

\itm Равносильность четырёх определений открытого множества на плоскости (через $B_r$, $\tilde B_r$, открытый и замкнутый прямоугольник). Доказательство кольцом.

\itm Предельные точки множества. Эквивалентное определение на плоскости (беск. много точек из $A$). $A'' = A'$\scolon $(\Cl A)' = A'$.

\itm $x \in \partial A$, $x \not \in A'$ $\Rightarrow$ $x \in A$. $x \in A'$, $x \not \in \partial A$ $\Rightarrow$ $x \in A$.

\itm Всюду плотные, коплотные, нигде не плотные, плотные в себе множества\scolon сепарабельные пространства. Вс.-пл. множество пересекается с любым открытым.

\itm $\b R$ сепарабельно. Вторая аксиома счётности $\Rightarrow$ сепарабельность. $\Int \Cl$ нигде не плотного множества.

\itm Непрерывное отображение. Пример: $f(x) = x^2$. Композиция непрерывных непрерывна, непрерывный образ сепарабельного пространства сепарабелен. Определение гомеоморфизма. Гомеоморфность — отношение эквивалентности.

\itm Гомеоморфность двух интервалов, интервала и прямой, $\b N$ и $\b Z$. Негомеоморфность отрезка и интервала, прямой и плоскости.

\itm Определение топологии прямого произведения через базу, проверка свойств базы. $(\b R^2, \O_{std})$ как топология прямого произведения $\b R \times \b R$. Определение проекций $\pr_1$, $\pr_2$, непрерывность проекций.

\itm Фактор-топология, факторное отображение: определения. Примеры: прямая с двумя началами, окружность\scolon стягивание в точку, приклеивание по отображению.

\itm Букет пространств. Склейки квадрата, проективная плоскость.

















\end{enumerate}
\end{document}