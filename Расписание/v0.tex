\documentclass[a4paper]{extarticle}
\pagestyle{empty}

\usepackage[12pt]{extsizes}
\usepackage[utf8x]{inputenc}
\usepackage[russian]{babel}

\usepackage[landscape,margin=1cm,top=1.5cm,bottom=2.15cm]{geometry}
\usepackage{makecell,fancyhdr,amsmath,multirow}
\usepackage{bigstrut}

\def\scolon{\rlap{,}\raisebox{0.8ex}{,} }


%%%%%%%%%%%%%%%%
%%%%%%%%%%%%%%%%


\pagestyle{fancy}
\renewcommand{\headrulewidth}{0mm}

\fancyhead{} \fancyfoot[CO,CE]{}
\fancyfoot[LO,LE]{\it\large Летняя математическая школа ЛНМО, Поставы, 2018 год}
\fancyfoot[RO,RE]{Typeset: \LaTeX.\quad\copyright\ Б.А., Д.Г., 2018
	\quad\copyright\ ЛНМО, 2018}


%%%%%%%%%%%%%%%%
%%%%%%%%%%%%%%%%


\def\vph{$\vphantom{\int\limits^{\int}}$}
\def\vphs{$\vphantom{\int\limits_0^0}$}

\def\para#1#2#3{
	\ \makecell[l]{\vph{\large\bf #1}
	\medskip \\ \it #2 \\ \it #3
	\medskip}
}

\def\nul{& \para{\ }{\ }{\ }}
\def\mfill#1{& \multicolumn{#1}{c|}{ }}

\begin{document}


%%%%%%%%%%%%%%%%
%%%%%%%%%%%%%%%%


\def\begn{
\vspace{0.8cm}
\begin{center}
\begin{tabular}{|l|l|l|l|l|l|}
\hline\vphs
	& \large 7 класс
	& \large 8 класс
	& \large 9 класс
	& \large 10 класс
	& \large 11 класс
}


\def\paraa{\\ \hline \makecell[l]{{\bf\Large 1} \medskip \\ 9:15–10:45}}
\def\parab{\\ \hline \makecell[l]{{\bf\Large 2} \medskip \\ 11:00–12:30}}
\def\parac{\\ \hline \makecell[l]{{\bf\Large 3} \medskip \\ 16:30–18:00}}
\def\parad{\\ \hline \makecell[l]{{\bf\Large 4} \medskip \\ 19:00–20:30}}\def\shdInfo{
	& \para
		{$\begin{array}{l}
			\text{\bf\small\!Д.Г.\,Штукенберг}
			\vspace{-1.4mm} \\
			\text{\bf\small\!М.А.\,Клеверов}
		\end{array}$}
		{Теоретическая}{информатика}}

\def\iaVved{& \para{И.А.\,Чистяков}{Введение}{в алгебру и анализ}}
\def\iaMath{& \para{И.А.\,Чистяков}{Алгебра}{и анализ}}
\def\iaIneq{& \para{И.А.\,Чистяков}{Неравенства}{\ }}

\def\zbaTopo{& \para{Б.А.\,Золотов}{Общая}{топология}}
\def\zbaTopoIm{& \para{Б.А.\,Золотов}{Общая}{топология$^\text{\rm\ \tiny(ИМ)}$}}
\def\zbaPerm{& \para{Б.А.\,Золотов}{Преобразования}{и перестановки}}
\def\zbaLalg{& \para{Б.А.\,Золотов}{Конечномерная}{линейная алгебра}}

\def\medSers{& \para{А.Н.\,Медведев}{Теория рядов}{}}
\def\medFan{& \para{А.Н.\,Медведев}{Метрические}{пространства}}
\def\medFanIm{& \para{А.Н.\,Медведев}{Метрические}
	{пространства$^\text{\rm\ \tiny(ИМ)}$}}

\def\alexGeom{& \para{И.С.\,Алексеев}{Введение в}{алг.\,геометрию}}
\def\alexGomo{& \para{И.С.\,Алексеев}{Теория}{гомотопий}}
\def\alexTopo{& \para{И.С.\,Алексеев}{Алгебраическая}{топология}}

\def\kuliOlym{
	& \para
		{$\begin{array}{l}
			\text{\bf\footnotesize\!П.А.\,Куликов}
			\vspace{-1.8mm} \\
			\text{\bf\footnotesize\!А.Н.\,Сердюков}
			\vspace{-1.8mm} \\
			\text{\bf\footnotesize\!И.М.\,Богданов}
		\end{array}$}
		{Математический}{кружок}
}

\def\lepesPhys{& \para{Ю.П.\,Лепескин}{Физика}{}}

\def\sobr{& \multicolumn{5}{l|}{\para{\hspace{2.6cm} Общее организационное собрание}
	{\hspace{2.6cm} Актовый зал,}
	{\hspace{2.6cm} главная аллея}}}


%%%%%%%%%%%%%%%%
%%%%%%%%%%%%%%%%


\def\kinoSmall{& \para{Киноклуб}{Актовый зал,}{главная аллея}}

\def\bogdMera{& \para{И.М.\,Богданов}{Теория меры}{}}
\def\bogdGraf{& \para{И.М.\,Богданов}{Теория графов}{}}

\vfill\eject
\noindent {\bf\LARGE Расписание занятий на \underline{22 июля, воскресенье}}

\begn
\paraa \sobr
\parab \iaVved \zbaPerm \lepesPhys \medSers \alexGomo
\parac \kuliOlym \lepesPhys \iaIneq \zbaLalg \medFan
\parad \kinoSmall \shdInfo \zbaTopo \lepesPhys \alexGomo
\\ \hline
\end{tabular}
\end{center}


%%%%%%%%%%%%%%%%
%%%%%%%%%%%%%%%%


\vfill\eject
\noindent {\bf\LARGE Распорядок дня \underline{22 июля, воскресенье}}

\large\vspace{1.2cm}
\begin{center} \begin{tabular}{ll}
\vphs \bf Когда\hspace{2.7cm} & \bf Что \\
8:00 & Подъём, обход врачей \\
9:20 & Завтрак \\
10:00—10:45 & Общее организационное собрание, актовый зал \\
11:15—12:45 & {\bf Первая пара занятий} \\
13:00 & Обед \\
13:30 & Медицинские процедуры, спортивные занятия \\
16:00 & Полдник (столовая) \\
16:15—17:45 & {\bf Вторая пара занятий} \\
18:15 & Ужин \\
18:45—20:15 & {\bf Третья пара занятий} \\
20:15—22:40 & Мероприятия, самоподготовка \\
23:00 & Отбой, полная тишина в корпусах \\
\end{tabular} \end{center}


%%%%%%%%%%%%%%%%
%%%%%%%%%%%%%%%%


\vfill\eject
\noindent {\bf\LARGE Расписание занятий c \underline{23 июля, понедельника}}

\begn
\paraa \iaVved \shdInfo \zbaTopo \lepesPhys \alexGomo
\parab \kuliOlym \lepesPhys \iaIneq \zbaLalg \medFan
\parac \shdInfo \iaMath \zbaTopoIm \medSers \alexTopo
\parad \nul \zbaPerm \lepesPhys \alexGeom \medFanIm
\\ \hline
\end{tabular}
\end{center}


%%%%%%%%%%%%%%%%
%%%%%%%%%%%%%%%%


\vfill\eject
\noindent {\bf\LARGE Распорядок дня с \underline{23 июля, понедельника}}

\large\vspace{1.2cm}
\begin{center} \begin{tabular}{ll}
\vphs \bf Когда\hspace{2.7cm} & \bf Что \\
8:00 & Подъём \\
8:30 & Завтрак \\
9:15—10:45 & {\bf Первая пара занятий} \\
11:00—12:30 & {\bf Вторая пара занятий} \\
13:00 & Обед \\
13:30—16:00 & Медицинские процедуры, спортивные занятия \\
16:00 & Полдник (столовая) \\
16:30—18:00 & {\bf Третья пара занятий} \\
18:30 & Ужин \\
19:00—20:30 & {\bf Четвертая пара занятий}, мероприятия \\
20:30—22:40 & Время для самоподготовки \\
23:00 & Отбой, полная тишина в корпусах \\
\end{tabular} \end{center}


%%%%%%%%%%%%%%%%
%%%%%%%%%%%%%%%%


\vfill\eject
\noindent {\bf\LARGE Расписание занятий c \underline{26 июля, четверга}}

\begn
\paraa \iaVved \shdInfo \zbaTopo \lepesPhys \alexGomo
\parab \shdInfo \iaMath \lepesPhys \zbaLalg \bogdMera
\parac \kuliOlym \zbaPerm \iaIneq \medSers \alexTopo
\parad \nul \lepesPhys \bogdGraf \alexGeom \medFan
\\ \hline
\end{tabular}
\end{center}


%%%%%%%%%%%%%%%%
%%%%%%%%%%%%%%%%


\vfill\eject
\noindent {\bf\LARGE Расписание занятий на \underline{28 июля, суббота}}

\begn
\paraa \iaVved \shdInfo \bogdGraf \lepesPhys \alexGomo
\parab \shdInfo \zbaPerm \lepesPhys \medSers \bogdMera
\parac \kuliOlym \lepesPhys &
	\para{ПОСТАВЫ}
		{Б.А.\,Золотов,}
		{Д.Г.\,Штукенберг}
	\alexGeom \medFan
\parad \nul \nul &
	\para{КИНО}
		{Актовый зал,}
		{гл.\,аллея}
	\zbaLalg \alexTopo
\\ \hline
\end{tabular}
\end{center}


%%%%%%%%%%%%%%%%
%%%%%%%%%%%%%%%%


\vfill\eject
\noindent {\bf\LARGE Расписание занятий на \underline{29 июля, воскресенье}}

\begn \\ \hline
{\bf\Large 1} & \multicolumn{3}{l|}{
	\multirow{2}*{\para{ГЛУБОКОЕ}
		{Поездка на автобусе | И.А., М.А., И.М., М.В.,}
		{С.В., П.А., Н.А. | Сбор у корпуса в 9:00}}
	}
	\medSers \alexGomo \\ \cline{1-1} \cline{5-6}
{\bf\Large 2} \mfill{3} \alexGeom \medFan \\ \hline
{\bf\Large 3} &
	\multicolumn{3}{l|}{\hspace{0.5in}
		\para{КИНО}{Актовый зал,}{главная аллея}}
	& \multicolumn{2}{c|}{\para{ОЗЕРО}
	{Пешком | Б.А., Д.Г., А.Н., Ю.П., А.Н., И.С. |\ }
		{Сбор у корпуса после обеда}} \\ \hline
{\bf\Large 4} \mfill{3}
		& \multicolumn{2}{c|}{\hspace{0.5in}
	\para{КИНО}{Актовый зал,}{главная аллея}} \\ \hline
\end{tabular}
\end{center}


%%%%%%%%%%%%%%%%
%%%%%%%%%%%%%%%%


\vfill\eject
\noindent {\bf\LARGE Рабочее расписание занятий c \underline{30 июля, понедельника}}

\begn
\paraa \iaVved \shdInfo \zbaTopo \lepesPhys \alexGomo
\parab \kuliOlym \lepesPhys \iaIneq \medSers \alexTopo
\parac \shdInfo \iaMath \bogdGraf \zbaLalg \medFan
\parad \nul \zbaPerm \lepesPhys \alexGeom \bogdMera
\\ \hline
\end{tabular}
\end{center}


%%%%%%%%%%%%%%%%
%%%%%%%%%%%%%%%%


\vfill\eject
\noindent {\bf\LARGE Расписание занятий на \underline{31 июля, вторник}}

\begn \\ \hline
{\bf\Large 1} \iaVved \shdInfo \zbaTopo & \multicolumn{2}{l|}{
		\multirow{3}*{\para{ПОЛОЦК}
			{Поездка на автобусе | Н.А., Ю.П., П.А.}
			{Сбор у корпуса №3 в 9:00}
	}} \\ \cline{1-4}
{\bf\Large 2} \kuliOlym \zbaPerm \iaIneq \mfill{2} \\ \cline{1-4}
{\bf\Large 3} \shdInfo \iaMath \bogdGraf \mfill{2} \\ \hline
{\bf\Large 4} & & \multicolumn{4}{c|}{
	\para{КИНО}{Актовый зал,}{главная аллея}}
\\ \hline
\end{tabular}
\end{center}


%%%%%%%%%%%%%%%%
%%%%%%%%%%%%%%%%


\vfill\eject
\noindent {\bf\LARGE Расписание занятий на \underline{4 августа, субботу}}

\begn
\paraa \iaVved \shdInfo \zbaTopo \medSers \bogdMera
\parab \shdInfo \zbaPerm \iaIneq \lepesPhys \medFan \\ \hline
\end{tabular}
\end{center}


%%%%%%%%%%%%%%%%
%%%%%%%%%%%%%%%%


\vfill\eject
\noindent {\bf\LARGE Расписание занятий на \underline{5 августа, воскресенье}}

\begn
\paraa &
\para{Экзамен по т.\,инф}{Д.Г.Ш., М.А.К.,}{9:00–11:00}
 \iaMath \zbaTopo &
	\para{Подготовка}{к экзамену}{по лин.\,алг.}
 \alexTopo
\parab \iaVved &
\para{Экзамен по т.\,инф}{Д.Г.Ш., М.А.К.,}{11:00–12:30}
 &
\para{Самоподготовка}{к зачёту}{по физике}
	& \para{Б.А.\,Золотов}{Лин.\,алг.}{(консультация)} \alexTopo
\parac & 
\para{Экзамен по т.\,инф}{Д.Г.Ш., М.А.К.,}{13:30–15:00}
&
\para{Экзамен по т.\,инф}{Д.Г.Ш., М.А.К.,}{15:00–18:00} &
\para{Ю.П.\,Лепескин}{Зачёт по физике}{c 16:00} &
\para{Б.А.\,Золотов}{Экз. по л.\,алг.}{с 16:00} \mfill{1} \\ \hline
\end{tabular}
\end{center}


%%%%%%%%%%%%%%%%
%%%%%%%%%%%%%%%%


\vfill\eject
\noindent {\bf\LARGE Расписание экзаменов ЛМШ ЛНМО, 2018}

\def\ekz#1#2{
	\ \makecell[l]{\vph{\large\bf #1}
	\medskip \\ \rm #2
	\medskip}}

\begn \\ \hline
6 августа
	& \ Пары / матбой
	& \ekz{Теория групп}{И.А.\,Чистяков}
	& \ekz{Теория графов}{И.М.\,Богданов}
	& \ekz{Физика}{Ю.П.\,Лепескин}
	& \ekz{Алг.\,топология}{И.С.\,Алексеев} \\ \hline
7 августа 
	& \ekz{Итог.\,олимпиада}{П.А.К., А.Н.С., И.М.Б.}
	& \ekz{Перестановки}{Б.А.\,Золотов}
	& \ekz{Неравенства}{И.А.\,Чистяков}
	& \ekz{Алг.\,геометрия}{И.С.\,Алексеев}
	& \ekz{Метр.\,пространства}{А.Н.\,Медведев} \\ \hline
8 августа
	& \ekz{Введение в мат.ан.}{И.А.\,Чистяков}
	& \ekz{Физика}{Ю.П.\,Лепескин}
	& \ekz{Топология}{Б.А.\,Золотов}
	& \ekz{Теория рядов}{А.Н.\,Медведев}
	& \ekz{Теория меры}{И.М.\,Богданов} \\ \hline
\makecell[l]{8 августа, \\ вечер}
	& \multicolumn{5}{l|}{\qquad\ekz{Заключительный концерт}{Актовый зал}} \\
	\hline

\end{tabular}
\end{center}

\ \\ [0.5cm] Время начала экзаменов по умолчанию — 9:15\scolon детали уточняйте у преподавателей.

\medskip\noindent О прочих мероприятиях (репетиции, матбои, экскурсии, $\ldots$) будет сообщаться отдельно.


%%%%%%%%%%%%%%%%
%%%%%%%%%%%%%%%%


\vfill\eject
\noindent {\bf\LARGE Расписание занятий на \underline{6 августа, понедельник}}

\def\podgGr{\para{Подготовка}{к экзамену}{по т.\,групп}}

\begn
\\ \hline
\makecell[l]{{\Large\bf 1} \\ 9:15} \iaVved
	& \multirow{2}*{\para{Подготовка}{к экзамену}{по алгебре}}
	& \multirow{2}*{\para{Подготовка}{к экзамену}{по т.\,графов}}
	& \multirow{2}*{\para{Подготовка}{к зачёту}{по физике}}
	& \multirow{2}*{\para{И.С.\,Алексеев}{Экзамен}{по алг.\,топологии}} \\
		\cline{1-2}
\makecell[l]{{\Large\bf 2} \\ 11:00} \kuliOlym & & & & \\ \hline
\makecell[l]{{\Large\bf 3} \\ 13:30} & \para{Матбой}{Подготовка,}{с 13:30}
	& \para{И.А.\,Чистяков}{Экз. по алгебре}{с 13:30}
	& \para{И.М.\,Богданов}{Экз. по т.\,графов}{с 13:30}
	& \para{Ю.П.\,Лепескин}{Зач. по физике}{с 13:30}
	& \multirow{2}*{\para{Подготовка}{к экзамену}{по метр.\,пр-вам}} \\
		\cline{1-5}
\makecell[l]{{\Large\bf 4} \\ 19:00} & \para{Матбой}{Бой,}{с 19:00}
	& \multicolumn{3}{l|}{\qquad\large\bf Самоподготовка} & \\ \hline

\end{tabular}
\end{center}


%%%%%%%%%%%%%%%%
%%%%%%%%%%%%%%%%


\vfill\eject
\noindent {\bf\LARGE Расписание занятий на \underline{7 августа, вторник}}

\begn \\ \hline
\makecell[l]{{\Large\bf 1} \\ 9:15}
	\iaVved
	& \para{Подготовка}{к экзамену}{по перестановкам}
	& \multirow{2}*{\para{Подготовка}{к экзамену}{по неравенствам}}
	& \multirow{3}*{\para{И.С.\,Алексеев}{Экзамен}{по алг.\,геом}}
	& \multirow{3}*{\para{А.Н.\,Медведев}{Экзамен}{по метр.\,пр-вам}}
		\\ \cline{1-3}
	
\makecell[l]{{\Large\bf 2} \\ 11:00}
	& \makecell[l]{ \\ {\bf\large\ Итоговая}
		\\ {\bf\large\ олимпиада}
		\\ {\it\ 11:00–13:00} \\ \ }
	& \multirow{2}*{\para{Б.А.\,Золотов}{Экзамен}{по перестановкам}}
	& & &
		\\ \cline{1-2} \cline{4-4}
	
\makecell[l]{\\ {\Large\bf 3} \\ 13:30 \\ \ }
	& \multirow{2}*{\para{Подготовка}{к экзамену}{по матану}}
	& & \para{И.А.\,Чистяков}{Экзамен}{по неравенствам} & &
		\\ \cline{1-1} \cline{3-6}
	
\makecell[l]{\\ {\Large\bf 4} \\ 19:00 \\ \ } 
	& & \multicolumn{4}{l|}{\qquad\bf\large Самоподготовка}\\ \hline

\end{tabular}
\end{center}


%%%%%%%%%%%%%%%%
%%%%%%%%%%%%%%%%


\vfill\eject
\noindent {\bf\LARGE Распорядок дня на \underline{8–9 августа}}

\large\vspace{1.2cm}
\begin{center} \begin{tabular}{ll}
\vphs \bf Когда\hspace{2.8cm} & \bf Что \\
8:00 & Подъём \\
8:30 & Завтрак \\
9:15 & {\bf Экзамены} \\
13:00 & Обед \\
13:30—16:00 & Завершение экзаменов, репетиции, спорт \\
16:00 & Полдник (столовая) \\
16:00—18:00 & Репетиции, сбор вещей \\
18:30 & Ужин \\
20:30—23:00 & Награждение учащихся, заключительный концерт\\
 & {\it\qquad (актовый зал)} \\
23:00—00:30 & Кино {\it (Э.\,Кустурица, «Завет»)} \\
00:30 & Отбой, полная тишина в корпусах \\
 & \\
{\bf 9 августа} & \\
07:00 & Подъём \\
07:30 & Завтрак \\
08:00 & Отъезд на автобусах \\
\end{tabular} \end{center}


%%%%%%%%%%%%%%%%
%%%%%%%%%%%%%%%%


\vfill\eject
\noindent {\bf\LARGE Разбиение на группы — 7 класс, экзамен И.А.\,Чистякова}

\vspace{1.5cm}

{\Large
\begin{center}
\begin{tabular}{l|l|l}
1 группа (9:15)\hspace{2cm}
	& 2 группа (10:30) \hspace{2cm}
		& 3 группа (11:45) \hspace{2cm} \\ \hline \hline
Борисенко &	Беляков			& Борисов
	$\vphantom{a^{\int\limits_0}}$\\
Бабий &		Гольдштейн		& Грищенко \\
Бородинова &	Денисова			& Дойникова \\
Бурлаков &		Иванов Александр	& Журавлева \\
Костин &		Мельников			& Иванов Андрей \\
Молчанов &	Муратов			& Корсунов \\
Ревякин &		Пономаренко		& Сороко \\
Рябикин &		Шишмарев			& Ковалев \\
Шутова &		Кубеков			& \\
\end{tabular}
\end{center}
}

\end{document}