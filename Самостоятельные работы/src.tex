{\noindent\bf Самостоятельная работа №9,\ \ 8 класс, перестановки}

\begin{itemize}
	\itm[$9$] Обратить перестановки: \vspace{-4mm}

		$$\begin{pmatrix}1 & 2 & 3 & 4 & 5 & 6 & 7 & 8 & 9 \\
			             9 & 7 & 4 & 1 & 6 & 3 & 5 & 2 & 8 \end{pmatrix}\scolon\ \ \ \ 
		\begin{pmatrix} 9&8&7&6&5 \end{pmatrix}\scolon\ \ \ \ 
		(a_1^{(1)}\,a_2^{(1)}) \circ (a_1^{(2)}\,a_2^{(2)}) \circ \ldots \circ (a_1^{(n)}\,a_2^{(n)}).$$
	\vspace{-4.5mm}

	\itm[$99$] Найти композиции $\sigma \circ \tau$ и $\tau \circ \sigma$, где\vspace{-4mm}

	$$\sigma = \begin{pmatrix} 1 & 2 & 3 & 4 & 5 & 6 & 7 & 8 & 9 \\
		9 & 1 & 2 & 3 & 4 & 5 & 6 & 7 & 8 \end{pmatrix}\scolon\ \ \ \ 
	\tau = \begin{pmatrix}1 & 2 & 3 & 4 & 5 & 6 & 7 & 8 & 9 \\
		9 & 6 & 2 & 7 & 4 & 8 & 5 & 1 & 3\end{pmatrix}.$$
	\vspace{-4.5mm}
	
Коммутируют ли эти перестановки? Опишите все перестановки, с которыми коммутирует $\sigma$.

	\itm[$99+9$] В условиях предыдущей задачи — на сколько непересекающихся циклов раскладывается перестановка $\sigma^n$ в зависимости от $n$?

	\itm[$0$.$9$] Найти сопряжённые перестановки $g^{-1}x_1g$ и $g^{-1}x_2g$, где\vspace{-7.5mm}
	
	$$g = \begin{pmatrix}1 & 2 & 3 & 4 & 5 & 6 & 7 & 8 & 9\\
		9&6&7&8&2&3&4&5&1\end{pmatrix}\scolon\ \ \ 
	x_1 = \begin{pmatrix}1 & 2 & 3 & 4 & 5 & 6 & 7 & 8 & 9\\
		9&4&7&8&6&1&5&2&3\end{pmatrix}\scolon\ \ \ 
	x_2=(9\;8\;7)(6\;5\;4)(3\;2\;1).$$
	\vspace{-5.5mm}
	
	\itm[$9^9$] Разложите на непересекающиеся циклы перестановку $x_1$ из задания $0.9$, а также перестановку \\ $\mu = (9\;5)(1\;2)(6\;8)(1\;3)(9\;7)(1\;4)$.

	\itm[$9\pm9$] Разложите на транспозиции перестановки $x_1$ и $x_2$ из задания 0.9.

	\itm[\it 9+\!+] Дано разложение перестановки $x$ на циклы. Как, не приводя $x$ к привычному виду, разложить на циклы перестановку $g^{-1} x g$?
	\itm[.]
\end{itemize}